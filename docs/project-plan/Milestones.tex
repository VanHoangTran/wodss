\chapter{Meilensteine}
\vspace{0.5cm}

\section{Meilenstein 1}
Mit Erreichung des ersten Meilensteins soll die Planung der Implementierung abgeschlossen sein. Resultierende Artefakte sind die folgenden: GUI Prototype, Technische Strukturierung (Klassendiagramm/Datenbankschema), Definition der Business-Logik sowie der benötigten bzw. angebotenen Schnittstellen.\\ \\
\textbf{Datum:} 26.03.2018

\section{Meilenstein 2}
Mit Erreichung des zweiten Meilensteins soll ein MVP (minimum viable product) implementiert sein, welcher die Basisfunktionalität umsetzt. Ebenfalls sollen mögliche Probleme und deren Lösungsvarianten zum genannten Zeitpunkt erkannt und evaluiert sein. Resultat ist eine Präsentation des Zwischenstandes. \\ \\
\textbf{Datum:} 09.04.2018

\section{Meilenstein 3}
Mit Erreichung des dritten Meilensteins sollen alle Anforderungen implementiert sein, das Produkt ist somit benutzbar. Resultierende Artefakte sind: Dokumentation der Lösung, Source-Code, Lauffähiges Deployment sowie die Schusspräsentation. Nach diesem Meilenstein werden keine neuen Funktionalitäten mehr implementiert.\\ \\
\textbf{Datum:} 21.05.2018

\section{Meilenstein 4}
Mit Erreichung des vierten und letzten Meilensteins soll das erarbeitete Produkt ausführlich getestet und wo nötig korrigiert worden sein.\\ \\
\textbf{Datum:} 28.05.2018